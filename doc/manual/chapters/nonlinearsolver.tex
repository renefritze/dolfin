\chapter{Nonlinear solver}
\index{nonlinear solver}

\dolfin{} provides tools for solving nonlinear equations of the form
\begin{equation}
  F(u) = 0, 
\end{equation}
where $F: \mathbb{R}^{n} \rightarrow \mathbb{R}^{n}$. 
To use the nonlinear solver, a nonlinear function must be defined. 
The nonlinear solver is then initialised with this function and a 
solution computed.



\section{Nonlinear functions}
\index{NonlinearFunction}

To solve a nonlinear problem, the user must defined a class which 
represents the nonlinear function $F(u)$. The class 
should be derived from the \dolfin{}  class \texttt{NonlinearFunction}. It
contains the necessary functions to form the function $F(u)$ and the 
Jacobian matrix  $J = \partial F / \partial u$. The precise form of the user 
defined class will depend on the problem being solved.
The structure of a user defined class \texttt{MyNonlinearFunction} is shown below.
%
\begin{code}
class MyNonlinearFunction : public NonlinearFunction
\{
  public: 
  
    // Constructor 
    MyNonlinearFunction() : NonlinearFunction()\{\}
  
    // Compute F(u) and J 
    void form(GenericMatrix& A, GenericVector& b, 
              const GenericVector& u)
    \{
      // Insert F(u) into the vector b and J into the matrix A 
    \}

  private:
    // Functions and pointers to objects with which F(u) is defined
\};
\end{code}
%
The above class computes the function $F(u)$ and its Jacobian $J$
concurrently. In the future, it will be possible to compute 
$F(u)$ and $J$ either concurrently or separately.



\section{Newton solver}
\index{Newton's method}
\index{NewtonSolver}
%
\dolfin{} provides tools to solve nonlinear systems using Newton's method
and variants of it. The following code solves a nonlinear problem, defined by 
\texttt{MyNonlinearFunction} using Newton's method.
%
\begin{code}
Vector u;
MyNonlinearFunction F;
NewtonSolver newton_solver;

nonlinear_solver.solve(F, x);
\end{code}
%

The maximum number of iterations before the Newton 
procedure is exited can be set through the \dolfin{} parameter
system, along with the relative and absolute 
tolerances the residual. This is illustrated below.
%
\begin{code}
  NewtonSolver nonlinear_solver;
  nonlinear_solver.set("Newton maximum iterations", 50);
  nonlinear_solver.set("Newton relative tolerance", 1e-10);
  nonlinear_solver.set("Newton absolute tolerance", 1e-10);
\end{code}

The Newton procedure is considered to have converged when
the residual $r_{n}$ at iteration $n$ is less than the 
absolute tolerance or the relative residual
$r_{n}/r_{0}$ is less than the relative tolerance.
By default, the residual at iteration $n$ is given
by
%
\begin{equation}
  r_{n} = \| F(u_{n}) \|.
\end{equation} 
%
Computation of the residual in this way can be set by
%
\begin{code}
NewtonSolver newton_solver;
newton_solver.set("Newton convergence criterion", "residual");
\end{code}

For some problems, it is more appropriate to consider changes in
the solution~$u$ in testing for convergence. At iteration $n$,
the solution is updated via
%
\begin{equation}
  u_{n} = u_{n-1} + du_{n} 
\end{equation}
%
where $du_{n}$ is the increment. When using an incremental
criterion for convergence, the `residual' is defined as
%
\begin{equation}
  r_{n} = \| du_{n} \|.
\end{equation} 
%
Computation of the incremental residual can be set by
%
\begin{code}
NewtonSolver newton_solver;
newton_solver.set("Newton convergence criterion", "incremental");
\end{code}
%


\subsection{Linear solver}
The solution to the nonlinear problems is returned in the vector~\texttt{x}.
By default, the \texttt{NewtonSolver} used a direct solver to solve systems
of linear equations. It is possible to set the type linear solver to be used 
when creating a \texttt{NewtonSolver}. For example, 
%
\begin{code}
NewtonSolver newton_solver(gmres);
\end{code}
%
creates a solver which will use GMRES to solve linear systems. For iterative
solvers, the preconditioner can also be selected,
\begin{code}
NewtonSolver newton_solver(gmres, ilu);
\end{code}
%
The above Newton solver will use GMRES in combination with incomplete LU 
factorisation.

\subsection{Application of Dirichlet boundary conditions}
%
The application of Dirichlet boundary conditions to finite element
problems in the context of a Newton solver requires particular 
attention. The `residual' $F(u)$ at nodes where Dirichlet boundary
conditions are applied is the equal to difference between the 
imposed boundary condition value and the current solution~$u$.
The function 
\begin{code}
void FEM::applyResidualBC(GenericVector& b, 
           const GenericVector& x, Mesh& mesh,
           FiniteElement& element, BoundaryCondition& bc)
\end{code}
applies Dirichlet boundary conditions correctly. For a nonlinear
finite element problem, the below code assembles the function $F(u)$
and its Jacobian, and applied Dirichlet boundary conditions in the
appropriate manner.
%
\begin{code}
class MyNonlinearFunction : public NonlinearFunction
\{
  public: 
  
    // Constructor 
    MyNonlinearFunction(. . . ) : NonlinearFunction(. . . )\{\}
  
    // Compute F(u) and J 
    void form(GenericMatrix& A, GenericVector& b, 
              const GenericVector& u)
    \{
      // Insert F(u) into the vector b and J into the matrix A 
      FEM::assemble(*a, *L, A, b, *_mesh);
      FEM::applyBC(A, *_mesh, a->test(), *_bc);
      FEM::applyResidualBC(b, x, *_mesh, a->test(), *_bc);
    \}

  private:
    // Functions and pointers to objects with which F(u) is defined
\};
\end{code}


\section{Incremental Newton solver}
%
Newton solvers are commonly used to solve nonlinear equations in a series 
of steps. This can be done by building a simple loop around a Newton solver,
and is shown in the following code.
%
\begin{code}
MyNonlinearProblem F(U);
NewtonSolver nonlinear_solver;

Vector& x = U.vector();

// Solve nonlinear problem in a series of steps
real dt = 1.0; real t  = 0.0; real T  = 3.0;
while( t < T)
\{
  t += dt;
  nonlinear_solver.solve(F, x);
\}
\end{code}
%
Typically, the boundary conditions and/or source terms will be dependent 
on~\texttt{t}.
 



