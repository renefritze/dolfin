\chapter{Nonlinear solver}
\index{nonlinear solver}

\devnote{This chapter is currently being written\ldots}

\dolfin{} provides tools for solving nonlinear equations of the form
\begin{equation}
  F(u) = 0
\end{equation}
where $F: \mathbb{R}^{n} \rightarrow \mathbb{R}^{n}$. The nonlinear solvers are
based on Newton's method and utilise functions from PETSc \cite{www:petsc}.    

To use the nonlinear solver, a nonlinear function must be defined. The nonlinear
solver is then initialised with this function and a solution computed.



\section{Nonlinear functions}
\index{NonlinearFunction}

To solve a nonlinear problem, the user must defined a class which . The class 
should be derived from the \dolfin class \texttt{NonlinearFunction}. The class 
should contain the necessary functions to form the function $F(u)$ and the 
Jacobian matrix  $J = \partial F / \partial u$. The precise form of the user 
defined class will depend on the PDE being solved and the numerical method.
The structu of a user defined class \texttt{MyNonlinearFunction} is shown below.
\begin{code}
class MyNonlinearFunction : public NonlinearFunction
\{
  public: 
  
    // Constructor 
    MyNonlinearFunction() : NonlinearFunction()\{\}
  
    // Compute F(u) 
    void F(Vector\& b, const Vector\& x)
    \{
      // Insert F(u) into the vector b 
    \}

    // Compute J
    void J(Matrix\& A, const Vector\& x)
    \{
      // Insert the Jacobian into the matrix A 
    \}

    dolfin::uint size()
    \{      
      // Return the dimension of the Jacobian matrix 
    \}

    dolfin::uint nzsize()
    \{      
      // Return the maximum number of zeroes per row of the Jacobian
    \}

  private:
    // Pointers to objects with which F(u) is defined
\};
\end{code}


\section{Newton solver}
\index{Newton's method}
\index{NewtonSolver}

\subsection{Linear solver}

\subsection{Application of Dirichlet boundary conditions}

The application of inhomogenuous Dirichlet boundary conditions in the context
of a Newton solver requires particular attention.



\subsection{Newton solver parameters}

\subsection{Application of Dirichlet boundary conditions}


\section{Incremental Newton solver}

