\chapter{The log system}

\fixme{This needs to be rewritten}

The purpose of the log system is to provide a simple and
clean interface for logging messages, including warnings
and errors.

The following functions / macros are provided for logging:

\begin{code}
  dolfin_info();
  dolfin_debug();
  dolfin_warning();
  dolfin_error();
  dolfin_assert();
\end{code}

Examples of usage:

\begin{code}
  dolfin_info("Created vector of size %d.", x.size());
  dolfin_debug("Opened file");
  dolfin_warning("Unknown cell type.");
  dolfin_error("Out of memory.");
  dolfin_assert(i < m);
\end{code}

Note that in order to pass additional arguments to the last
three functions (which are really macros, in order to
automatically print information about file names, line numbers
and function names), the variations \texttt{dolfin\_debug1()},
\texttt{dolfin\_debug2()} and so on, must be used.

As an alternative to \texttt{dolfin\_info()}, C++ style output to cout
(\texttt{dolfin::cout}, and not \texttt{std::cout}) can be used. These messages
will be delivered to the same destination as messages by use
of the function \texttt{dolfin\_info()}.

Examples of usage:

\begin{code}
  cout << "Assembling matrix: " << A << endl;
  cout << "Refining grid: " << grid << endl;
\end{code}

The \texttt{dolfin\_assert()} macro should be used for simple tests that
may occur often, such as checking indexing in vectors. The check
is turned on only if DEBUG is defined.

To notify progress by a progress session, use the class
Progress.

Examples of usage:

\begin{code}
  Progress p("Assembling", grid.noCells());
  
  for (CellIterator c(grid); !c.end(); ++c) {
    ...
    p++;
  }
\end{code}

Progress also supports the following usage:

\begin{code}
  p = i;    // Specify step number
  p = 0.5;  // Specify percentage
  p.update(t/T, "Time is t = %f", t);
\end{code}
